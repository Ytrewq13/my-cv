%! TEX root = ./cv.tex
\documentclass[cv.tex]{subfiles}
\begin{document}
    \title{\href{https://github.com/Ytrewq13}{\texttt{https://github.com/Ytrewq13}}}
    \author{Sam Robert Whitehead}

    \maketitle

    \pagestyle{empty}
    \thispagestyle{empty}

\begin{xen}
    Tel:
\end{xen}
\begin{xcn}
    电话:
\end{xcn}
    (+44)7478190577
    \hfill
\begin{xen}
    Email:
\end{xen}
\begin{xcn}
    电子邮件:
\end{xcn}
    sam.everythingcomputers@gmail.com\\
    \vspace{1mm}
\begin{xen}
    Currently in my fourth year studying Computer Science at the University of
    Nottingham. Interested in all aspects of software development, with
    particular interest in system administration and operating system design.
\end{xen}
\begin{xcn}
    我现在就读于诺丁汉大学,是一名大四的学生,主要学习软件开发和编程。
\end{xcn}
\begin{xen}
    \section{Education}
\end{xen}
\begin{xcn}
    \section{教育}
\end{xcn}
\begin{xen}
    \subsection{University of Nottingham
        \hfill 2018-09 -- current}
\end{xen}
\begin{xcn}
        \subsection{诺丁汉大学
        \hfill 2018年09月 --- 现在}
\end{xcn}
\begin{xen}
        Degree: MSci Computer Science
        \hfill
        Average grade: 76\% (4.0 GPA equivalent)
\end{xen}
\begin{xcn}
        学士,专业:计算机科学
        \hfill
        GPA: 4.0
\end{xcn}
        \begin{itemize}
\begin{xen}
            \item \emph{Algorithms and data structures} in C, Java, C++,
                Python, and Haskell.
\end{xen}
\begin{xcn}
            % Algorithms and data structures in C, Java, C++, Python, and
            % Haskell.
            \item C、Java、C++、Python 和 Haskell 中的\emphcn{算法和数据结构}。
\end{xcn}
\begin{xen}
            \item Object Oriented Programming (OOP).
\end{xen}
\begin{xcn}
            % Object Oriented Programming (OOP).
            \item 面向对象编程。
\end{xcn}
\begin{xen}
            \item Functional Programming.
\end{xen}
\begin{xcn}
            % Functional Programming.
            \item 函数式编程。
\end{xcn}
\begin{xen}
            \item \emph{Compiler design} and implementation.
\end{xen}
\begin{xcn}
            % Compiler design and implementation.
            \item \emphcn{编译器设计}与实现。
\end{xcn}
\begin{xen}
            \item Robotics programming.
\end{xen}
\begin{xcn}
            % Robotics programming.
            \item 机器人编程。
\end{xcn}
\begin{xen}
            \item Information \emph{Visualization}, Data modeling and
                \emph{Analysis}.
\end{xen}
\begin{xcn}
            % Information Visualization, Data modeling and Analysis.
            \item 信息\emphcn{可视化}、数据建模与\emphcn{分析}。
\end{xcn}
\begin{xen}
            \item Dissertation: \emph{Operating System development} (in
                progress).
\end{xen}
\begin{xcn}
            \item 论文:\emphcn{操作系统开发}(进行中)。
\end{xcn}
        \end{itemize}
\begin{xen}
    \subsection{NEW College Pontefract
        \hfill 2016-09 -- 2018-07}
        \textbf{A-levels}: Achieved 3 A-Levels, with grades A*AA.
%           \begin{tabular}{cccc}
%               \textbf{Subject} & \textbf{Exam Board} & \textbf{AS/A2} &
%               \textbf{Grade}\\
%               Mathematics & AQA & A2 & A* \\
%               Further Mathematics & AQA & A2 & A \\
%               Computer Science & OCR & A2 & A \\
%               Physics & AQA & AS & A
%           \end{tabular}
\end{xen}
\begin{xcn}
        \subsection{新大学Pontefract
            \hfill 2016年09月 --- 2018年07月}
            \textbf{A-levels}: 实现了3个A-Level,成绩是A*AA
\end{xcn}
%\begin{xen}
%   \subsection{Kettlethorpe High School
%       \hfill 2011-09 -- 2016-07}
%       Achieved 10 GCSEs, including 5 A* grades and 5 A grades.
%       upon request).
%\end{xen}
%\begin{xcn}
%  \subsection{Kettlethorpe 中学
%      \hfill 2011年09月 --- 2016年07月}
%      有5个A*成绩和5个A成绩
%\end{xcn}
\begin{xen}
%\section{Experience}
\section{Projects}
\end{xen}
\begin{xcn}
\section{项目}
\end{xcn}
        \begin{itemize}
\begin{xen}
            \item \href{https://github.com/Ytrewq13/mandelbrotc}{Mandelbrot generator in C}
\end{xen}
\begin{xcn}
            \item \href{https://github.com/Ytrewq13/mandelbrotc}{曼的布罗特渲染器的C}
\end{xcn}
\begin{xen}
    \begin{itemize}
        \item \emph{Libpng} for storing images.
        \item Complex arithmetic and \emph{multithreading} in C.
    \end{itemize}
\end{xen}
\begin{xcn}
    -- \emphcn{Libpng}用于在C中存储图像,\emphcn{复杂的算术}和\emphcn{多线程}
\end{xcn}
\begin{xen}
            \item \href{https://github.com/Ytrewq13/mazemaker.git}{Maze generator and solver in Java and Javascript}
\end{xen}
\begin{xcn}
            \item \href{https://github.com/Ytrewq13/mazemaker.git}{迷宫发生器和求解器的Java和Javascript}
\end{xcn}
\begin{xen}
    \begin{itemize}
        \item \emph{Optimisation} of Javascript rendering code.
        \item A* search algorithm.
    \end{itemize}
\end{xen}
\begin{xcn}
    -- JavaScript呈现功能的\emphcn{优化},A *搜索算法
\end{xcn}
\begin{xen}
            \item Major \emph{contributor} to
                \href{https://github.com/nsxiv/nsxiv.git}{nsxiv} (Revival of
                image viewer \href{https://github.com/muennich/sxiv}{sxiv})
\end{xen}
\begin{xcn}
            \item \href{https://github.com/nsxiv/nsxiv.git}{nsxiv} 的主要贡献者(图像查看器 \href{https://github.com/muennich/sxiv}{sxiv} 的复兴)
\end{xcn}
\begin{xen}
\begin{itemize}
    \item Implemented loading for animated WebP images.
    \item Learned Google's WebP API and sxiv's pre-existing codebase to solve a
        real-world problem with an \emph{open-source software project}.
    \item Performed ongoing maintenance by fixing bugs and reviewing Pull
        Requests.
\end{itemize}
\end{xen}
\begin{xcn}
    -- 学习了\emphcn{Google的WebP API}和sxiv的\emphcn{预先存在的代码库},
    以通过\emphcn{开源软件项目}解决\emphcn{现实世界的问题}
\end{xcn}
        \end{itemize}
\begin{xen}
\section{Additional experience}
\end{xen}
\begin{xcn}
\section{经历}
\end{xcn}
\begin{xen}
    \subsection{Software Engineering Group project -- designated ``gitmaster''
    \hfill 2019-09 -- 2020-06}
        \begin{itemize}
            \item Organised the \emph{git} repository for the project's source
                code - Advanced \emph{version control}.
            \item Managed the ideas of a diverse group of people to create a
                \emph{cohesive software solution} to an industry challenge.
        \end{itemize}
\end{xen}
\begin{xcn}
    \subsection{软件工程小组项目 -- \emphcn{版本控制}主管
    \hfill 2019年09月 -- 2020年06月}
        \begin{itemize}
            \item 为该项目的\emphcn{git}存储库进行了组织
                源代码-高级\emphcn{版本控制}。
            \item 管理了一个\emphcn{多样化群体}的想法,
                创建一个\emphcn{内聚软件解决方案} \emphcn{行业挑战}。
        \end{itemize}
\end{xcn}
%\begin{xen}
%    \subsection{Personal Programming Projects}
%        \hspace{1em} \textbf{Project description}
%\end{xen}
%\begin{xcn}
%    \subsection{个人编项目} % Personal Programming Projects
%        \hspace{1em} \textbf{项目介绍} % Project description
%\end{xcn}
%        \hfill
%        \texttt{https://github.com/Ytrewq13/...}
\begin{xen}
\section{Programming Languages}
\end{xen}
\begin{xcn}
\section{编程语言}
\end{xcn}
\langa{
    C,
    Bash,
    Haskell,
    Python,
    \LaTeX,
}\langb{
    R (ggplot2),
    AWK,
    SED,
    VimL,
}\langc{
    Groff,
    C++,
    Perl,
}\langd{
    SQL,
    Javascript,
    Java,
}\lange{
    Lua,
    PHP,
    C\#,
    Agda,
    Coq,
    \ldots
}
\begin{xen}
\section{Technologies}
\end{xen}
\begin{xcn}
\section{技术}  % Technologies
\end{xcn}
\langa{
    Vim,
    Linux,
    Git,
    SSH,
}\langb{
    Systemd,
    Clang/LLVM,
    CMake,
    GDB,
}\langc{
    Ansible,
    Proxmox VE,
    QEMU/KVM,
    ZFS,
    RAID,
}\langd{
    ARM architectures,
    Make,
}\lange{
    Email (postfix, dovecot, OpenDKIM),
    \ldots
}
%\begin{xen}
%\section{Languages}
%\end{xen}
%\begin{xcn}
%\section{语言}
%\end{xcn}
%\begin{xen}
%    \begin{itemize}
%        \item English (Native speaker)
%        \item Chinese - Mandarin (Learning)
%    \end{itemize}
%\end{xen}
%\begin{xcn}
%    \begin{itemize}
%        \item 英语(母语人士的)
%        \item 汉语(在学习)
%    \end{itemize}
%\end{xcn}
\end{document}
