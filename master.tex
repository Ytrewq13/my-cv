%! TEX root = ./cv.tex
    \title{\href{https://github.com/Ytrewq13}{\texttt{https://github.com/Ytrewq13}}}
    \author{Sam Robert Whitehead}

    \maketitle

    \pagestyle{empty}

\begin{xen}
    Tel:
\end{xen}
\begin{xcn}
    电话:
\end{xcn}
    +447478190577
    \hfill
\begin{xen}
    Emails:
\end{xen}
\begin{xcn}
    电子邮件:
\end{xcn}
    sam.everythingcomputers@gmail.com\\
    {\raggedleft{psysrw@nottingham.ac.uk\par}}
    \vspace{1mm}
\begin{xen}
    Currently in my third year studying Computer Science at the University of
    Nottingham. Interested in all aspects of software development, with
    particular interest in system administration and writing performant,
    efficient code through various optimisations.
\end{xen}
\begin{xcn}
    我现在是大学三年级的学生。我在诺丁汉大学学习。我对软件开发兴趣,编程。
\end{xcn}
\begin{xen}
    \section{Education}
\end{xen}
\begin{xcn}
    \section{教育}
\end{xcn}
\begin{xen}
    \subsection{University of Nottingham
        \hfill 2018-09 -- current}
\end{xen}
\begin{xcn}
        \subsection{诺丁汉大学
        \hfill 2018年09月 --- 现在}
\end{xcn}
\begin{xen}
        Degree: MSci Computer Science
        \hfill
        First year grade: First\\
        {\raggedleft Second year grade: First\par}
\end{xen}
\begin{xcn}
        学士,专业:计算机科学
        \hfill
        等一年等级:Distinction\\
        {\raggedleft{等二年等级:Distinction\par}}
\end{xcn}
\begin{xen}
        \begin{itemize}
            \item Experience implementing various \emph{algorithms and data
                structures} in C.
            \item Modules include: \emph{Object Oriented Programming},
                Functional Programming (Haskell), \emph{Compiler design} and
                implementation.
        \end{itemize}
\end{xen}
\begin{xcn} % TODO: Update this
        \begin{itemize}
            \item 各种算法和数据结构 (C语言).
            \item 课程:
面向对象编程,
功能编程 (Haskell) \textendash{} 基本面和先进的技术,
软件工程小组项目,
编译器设计
        \end{itemize}
\end{xcn}
\begin{xen}
    \subsection{NEW College Pontefract
        \hfill 2016-09 -- 2018-07}
        \subsubsection{A Levels}
            Achieved 3 A Levels, with grades A*AA.
%           \begin{comment}
%               \begin{tabular}{cccc}
%                   \textbf{Subject} & \textbf{Exam Board} & \textbf{AS/A2} &
%                   \textbf{Grade}\\
%                   Mathematics & AQA & A2 & A* \\
%                   Further Mathematics & AQA & A2 & A \\
%                   Computer Science & OCR & A2 & A \\
%                   Physics & AQA & AS & A
%               \end{tabular}
%           \end{comment}
\end{xen}
\begin{xcn}
        \subsection{新大学Pontefract
            \hfill 2016年09月 --- 2018年07月}
            \subsubsection{A Levels}
            实现了3个A-Level,成绩是A*AA
\end{xcn}
%\begin{xen}
%   \begin{comment}
%   \subsection{Kettlethorpe High School
%       \hfill 2011-09 -- 2016-07}
%       Achieved 10 GCSEs, including 5 A* grades and 5 A grades.
%       upon request).
%   \end{comment}
%\end{xen}
%\begin{xcn}
%  \begin{comment}
%  \subsection{Kettlethorpe 中学
%      \hfill 2011年09月 --- 2016年07月}
%      有5个A*成绩和5个A成绩
%  \end{comment}
%\end{xcn}
\begin{xen}
\section{Experiences}
\end{xen}
\begin{xcn}
\section{经历}
\end{xcn}
\begin{xen}
    \subsection{Teamwork experience}
\end{xen}
\begin{xcn}
    \subsection{团队合作经验}
\end{xcn}
\begin{xen}
        \subsubsection{Student Exec Treasurer at NEW College Pontefract
        \hfill 2017-09 -- 2018-07}
                \emph{Tracked Student Exec funding} for the year, and
                \emph{allocated budget} to Exec projects.
\end{xen}
\begin{xcn}
        \subsubsection{新大学Pontefract的学生主管司库 % Student Exec Treasurer at NEW College Pontefract
        \hfill 2017年09月 -- 2018年07月}
                \emphcn{跟踪了年度支出}
                \emphcn{为项目分配了预算}
\end{xcn}
\begin{xen}
        \subsubsection{Software Engineering Group project -- designated \emph{gitmaster}
        \hfill 2019-09 -- 2020-06}
            \begin{itemize}
                \item Organised the \emph{git} repository for the project's
                    source code - Advanced \emph{version control}.
                \item Managed the ideas of a \emph{diverse group} of people to
                    create a \emph{cohesive software solution} to an
                    \emph{industry challenge}.
%                \item TODO: how to write teamwork skills in a positive way
            \end{itemize}
\end{xen}
\begin{xcn}
        \subsubsection{软件工程小组项目 -- \emphcn{版本控制}主管
        \hfill 2019年09月 -- 2020年06月}
            \begin{itemize}
                \item 为该项目的\emphcn{git}存储库进行了组织
                    源代码-高级\emphcn{版本控制}。
                \item 管理了一个\emphcn{多样化群体}的想法,
                    创建一个\emphcn{内聚软件解决方案} \emphcn{行业挑战}。
            \end{itemize}
\end{xcn}
\begin{xen}
    \subsection{Personal Programming Projects}
        \hspace{1em} \textbf{Project description}
\end{xen}
\begin{xcn}
    \subsection{个人编项目} % Personal Programming Projects
        \hspace{1em} \textbf{项目介绍} % Project description
\end{xcn}
        \hfill
        \texttt{https://github.com/Ytrewq13/...}
        \begin{itemize}
\begin{xen}
            \item Mandelbrot generator in C
\end{xen}
\begin{xcn}
            \item 曼的布罗特渲染器的C
\end{xcn}
                \hfill
                \href{https://github.com/Ytrewq13/mandelbrotc.git}{\texttt{mandelbrotc.git}}
\begin{xen}
    \\ \emph{Libpng} for storing images, \emph{complex arithmetic} and
    \emph{multithreading} in C.
\end{xen}
\begin{xcn}
    \\ \emphcn{Libpng}用于在C中存储图像,\emphcn{复杂的算术}和\emphcn{多线程}
\end{xcn}
\begin{xen}
            \item Maze generator and solver in Java and Javascript
\end{xen}
\begin{xcn}
            \item 迷宫发生器和求解器的Java和Javascript
\end{xcn}
                \hfill
                \href{https://github.com/Ytrewq13/mazemaker.git}{\texttt{mazemaker.git}}
\begin{xen}
    \\ \emph{Optimisation} of Javascript rendering code, \emph{A* search}
    algorithm.
\end{xen}
\begin{xcn}
    \\ JavaScript呈现功能的\emphcn{优化},A *搜索算法
\end{xcn}
\begin{xen}
            \item WebP image loader in C (extension of sxiv image viewer)
\end{xen}
\begin{xcn}
            \item C中的WebP图像加载器(sxiv图像查看器的扩展)
\end{xcn}
                \hfill
                \href{https://github.com/Ytrewq13/sxiv.git}{\texttt{sxiv.git}}
\begin{xen}
    \\ Learned \emph{Google's WebP API} and sxiv's \emph{pre-existing codebase}
    to solve a \emph{real-world problem} with an \emph{open-source software
    project}.
\end{xen}
\begin{xcn}
    \\ 学习了\emphcn{Google的WebP API}和sxiv的\emphcn{预先存在的代码库},
    以通过\emphcn{开源软件项目}解决\emphcn{现实世界的问题}
    % TODO: VERIFY THESE TRANSLATIONS - THEY ARE ALMOST CERTAINLY WRONG
\end{xcn}
        \end{itemize}
\begin{xen}
\section{Programming Languages}
\end{xen}
\begin{xcn}
\section{编程语言}
\end{xcn}
\langa{
    C,
    Bash,
    Haskell,
    Python,
    \LaTeX,
    Makefile,
}\langb{
    CMake,
    R (ggplot2),
    AWK,
    SED,
    Vimscript,
}\langc{
    Groff (MS macros, EQN, REF),
    C++,
    Perl,
}\langd{
    SQL,
    Javascript,
    Java,
}\lange{
    Lua,
    PHP,
    C\#,
    Agda,
    Coq,
    \ldots
}
\begin{xen}
\section{Languages}
\end{xen}
\begin{xcn}
\section{语言}
\end{xcn}
\begin{xen}
    \begin{itemize}
        \item English (Native speaker)
        \item Chinese - Mandarin (Learning)
    \end{itemize}
\end{xen}
\begin{xcn}
    \begin{itemize}
        \item 英语(母语人士的)
        \item 汉语(在学习)
    \end{itemize}
\end{xcn}
