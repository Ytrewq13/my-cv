% !TEX program = xelatex

\documentclass[10pt]{extarticle}

\usepackage{titlesec}
\usepackage{titling}
\usepackage[margin=0.88in]{geometry}
\usepackage{multicol}
\usepackage[inline]{enumitem}
\usepackage{parskip}
\usepackage{verbatim}
\usepackage{xeCJK}

\setCJKmainfont{HanaMinA}

\setlist{nosep}

\renewcommand{\maketitle}{
    {
    \centering
        \huge\bfseries\theauthor{}\\
        \Large\thetitle{}\par
    }
}

\titleformat{\section}
{\bfseries\Large}
{}
{0em}
{}[\titlerule]

\titleformat{\subsection}
{\bfseries\large}
{}
{0em}
{}

\titleformat{\subsubsection}
{\bfseries}
{}
{0em}
{}

\begin{document}
    \title{\texttt{https://github.com/Ytrewq13}}
    \author{Sam Robert Whitehead}

    \maketitle

    \pagestyle{empty}

    电话: +447478190577
    \hfill
    电子邮件: sam.everythingcomputers@gmail.com\\
    {\raggedleft{psysrw@nottingham.ac.uk\par}}
    \vspace{1mm}
    我现在是三年的大学学生。我学习在诺丁汉大学。我跟软件开发兴趣,特别高效的编程。
%   Currently in my third year studying Computer Science at the University of
%   Nottingham. Interested in all aspects of software development, with
%   particular interest in system administration and writing performant,
%   efficient code through various optimisations.
    \section{教育} % Education
        \subsection{诺丁汉大学 % University of Nottingham
        \hfill 2018年09月 --- current}
        硕士,专业:计算机科学
%       Degree: MSci Computer Science
        \hfill
        一年的年级:等级\\
%       First year grade: First\\
        {\raggedleft{二年的年级:等级\par}}
%       {\raggedleft Second year grade: First\par}
        \begin{itemize}
            \item 各种算法和数据结构 (C程式语言).
%           \item Experience implementing various algorithms and data structures in
%               C.
            \item 课程:
%           \item Modules so far include:
                \begin{itemize}
                    \item 面向对象编程
%                   \item Object Oriented Programming
                    \item 功能编程 (Haskell) \textendash{} 基本面和先进的技术
%                   \item Functional Programming (Haskell) \textendash{}
%                       Fundamentals \& Advanced Techniques
                    \item 软件工程小组项目
%                   \item Software Engineering Group Project
                    \item 编译器设计
%                   \item Compiler design and implementation
                \end{itemize}
            \item 可根据要求提供完整清单
%           \item Can provide a comprehensive list of modules upon request
%               (including grades)
        \end{itemize}
        \subsection{新大学Pontefract % NEW College Pontefract
            \hfill 2016年09月 --- 2018年07月}
            \subsubsection{A Levels}
            实现了3个A-Level,成绩是A*AA(可根据要求提供完整清单)
%           Achieved 3 A Levels, with grades A*AA (can provide details upon request).
            \begin{comment}
                \begin{tabular}{cccc}
                    \textbf{Subject} & \textbf{Exam Board} & \textbf{AS/A2} &
                    \textbf{Grade}\\
                    Mathematics & AQA & A2 & A* \\
                    Further Mathematics & AQA & A2 & A \\
                    Computer Science & OCR & A2 & A \\
                    Physics & AQA & AS & A
                \end{tabular}
            \end{comment}
        \subsection{Kettlethorpe 中学 % Kettlethorpe High School
            \hfill 2011年09月 --- 2016年07月}
            实现了10个GCSE,有5个A*成绩和5个A成绩(可根据要求提供完整清单)
%           Achieved 10 GCSEs, including 5 A* grades and 5 A grades (can provide details
%           upon request).
    \section{经历} % Experiences
    \subsection{个人编程项目} % Personal Programming Projects
    \hspace{1em} \textbf{项目介绍} % Project description
    \hfill
    \texttt{https://github.com/Ytrewq13/\ldots}
    \begin{itemize}
        \item 曼的布罗特渲染器德C
%       \item Mandelbrot generator in C
            \hfill
            \texttt{mandelbrotc.git}
        \item 堆排序德C
%       \item Heapsort in C
            \hfill
            \texttt{heapsort.git}
        \item 魔方计时器德Java
%       \item Rubik's cube timer in Java
            \hfill
            (A Level 课程作业)
%           (A Level Coursework)
            \hfill
            \texttt{cubetimerapplicationjavafx.git}
        \item 迷宫发生器和求解器德Java和Javascript
%       \item Maze generator and solver in Java and Javascript
            \hfill
            \texttt{mazemaker.git}
        \item 弹跳秋的物理模拟德Javascript
%       \item Bouncing ball physics simulation in Javascript
            \hfill
            \texttt{javascript-bouncing-balls.git}
        \item 遗传算法实例德Python和Javascript
%       \item Genetic algorithm example program in Python and Javascript
            \hfill
            \texttt{wordevolution.git}
    \end{itemize}
    \section{编程语言} % Programming Languages
    \subsection{流利的} % Very comfortable with
    C, Java, Python, Javascript, Bash, \LaTeX{}
    \subsection{工作熟悉} % Working familiarity
    C++, R (ggplot2), PHP, Lua, C\#, Vimscript, Groff (MS macros, EQN, REF),
    AWK, SED, Perl, Haskell, SQL

\end{document}
